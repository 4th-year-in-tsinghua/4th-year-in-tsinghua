\chapter{工字厅}

当晚我们离开西湖时,时间已近深夜,于是我们准备直接返回寝室。
骑车从工字厅前路过时,我们发现工字厅大门前还亮着灯。昏黄的灯光照射在紧闭的红色大门上,显得格外诡异。
看到这种景象,我不由得想起以前看过的一个鬼故事。
说是故宫即使在昼长夜短的夏季也要在下午5点之前就闭馆,原因在于,日月开始交接之后,天地间阳气渐稀,阴气渐盛,一些白日里不敢见光的东西就会开始出来活动。
而故宫作为明清两代的皇宫,四百多年来前廷后宫之中不知发生了多少冤疑悬案,于是一到黄昏时分,出来活动的厉鬼冤魂想来是不计其数,因此为了游客的安全,故宫总要保证在太阳落山之前就要闭馆谢客。

本着不能只有我一个人担惊受怕的想法,我把这个鬼故事讲给了C君,结果C君竟回敬了我一个关于工字厅的鬼故事。
鉴于我们平时就生活在清华园内,这样一个故事某种程度上更加惊悚。

据说那时工字厅门前还不在夜里点灯。
有一次,某位老师下班后又加了很久的班,等他忙完要离开时,天已经完全黑了。
他匆匆收拾好东西就出了门,准备离开。要去开车时,他发现自己竟把车钥匙落在了办公室中,于是又回去拿车钥匙。
然而他这趟回来,跨进工字厅的大门后,却发现原来的办公室不见了,门后是一个从未见过的世界。
一排排硬山顶的传统民房沿几条平行的街道排放,稍远的地方还有二层、三层的阁楼,有些楼下还搭着台子,台子上说书的、唱曲的、献舞的都兴奋地表演着。
街上户户灯火通明,好不红火!
大街上满是没见过的奇奇怪怪的人,有人带着奇怪的面具,有人竖着狐狸耳朵,还有人用袍子将自己整个罩住,只露出一张脸。
虽然各种奇奇怪怪的人都有,但所有人都说笑着在街市上游玩。
街道两旁有卖小吃的,烧鸡酱鸭卤鹅,羊腿猪耳臭豆腐,烧饼凉粉水馒头,各色齐全;
还有各种各样的小玩意儿,没有风也能转的风车,自己发光的石头,变大变小的帽子,无奇不有。
街上人头攒动,除了逛着玩的,还有跑出店来拉客的,殷勤地向游人发放着大红纸印制的广告,诉说着店里的各种妙处。
他愣了一下,急忙要退出来,就发现一个身影飞快地从自己身边飞过,穿出大门而去了。
他不禁吓了一跳,抬眼一看,才发现原来还有人飞在空中。
空中飞着的人身影显得有些缥缈,有些透明,似实似幻,仿佛下一秒就会消散一样,而且越向下接近脚的地方就越透明、越模糊,许多人的脚如果不定睛仔细去看,就已完全看不到了。
再向远方的空中去瞧,隐约间似乎有个岛浮在空中,只是仿佛隔着一层雾一样,实在是看不真切。
这位老师此时已深切地意识到,自己这次是闯入了别人的世界,心中不由得紧张起来,不知道自己还能不能平安回去。想到刚才穿门而出的飞灵,他也不由得猜测这门便是联通两个世界的通道。
想罢,他便赶紧仍从大门退出去,发现工字厅果然还是那个工字厅,清华园果然还是那个清华园;
心中有了点底气,抱着试一试的心态,他再一次跨进大门,果然还是那个光怪陆离的异世界,再退出来,便仍旧是深夜已临的清华园。
这道门,确实已变成了现世与另一个世界的通道。
这位老师心中颇有些感慨难以言明,无处抒发,只是忙从旁边的小侧门绕进工字厅取了钥匙走人了,并暗自发誓从此再也不在工字厅加班。
而就是从这之后,夜里工字厅门前点起了灯,老师们出入也开始只走旁边的小门。
坊间传闻是这位老师在校务会上提出了此事,于是本着对怪力乱神敬而远之的想法,工字厅形成了这样的惯例。

C君讲罢之时,我们正好骑到二教旁边。
二教平日里便不开放自习,深夜更是整座楼都黑不隆咚的,如今外面还围了一圈围挡,更是显得凄凉。
受这氛围所动,我又一下子联想起流传甚广的二教鬼故事,身上不禁升起一阵恶寒,于是催促C君迅速离开了这片是非之地。

\vfill

\paragraph{记事}
6月23日,笔者与C君寻得零零阁后,准备去紫操再叙。骑车途径工字厅,发现工字厅大门在晚上还亮着灯。灯光昏黄,显得有一点点阴森和神秘。
