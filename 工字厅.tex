\chapter{工字厅}

当晚我们离开西湖时,时间已近深夜,于是我们准备直接返回寝室。
骑车从工字厅前路过时,我们发现工字厅大门前还亮着灯。昏黄的灯光照射在紧闭的红色大门上,显得格外诡异。
看到这种景象,我不由得想起以前看过的一个鬼故事。
说是故宫即使在昼长夜短的夏季也要在下午5点之前就闭馆,原因在于,日月开始交接之后,天地间阳气渐稀,阴气渐盛,一些白日里不敢见光的东西就会开始出来活动。
而故宫作为明清两代的皇宫,四百多年来前廷后宫之中不知发生了多少冤疑悬案,于是一到黄昏时分,出来活动的厉鬼冤魂想来是不计其数,因此为了游客的安全,故宫总要保证在太阳落山之前就要闭馆谢客。

本着不能只有我一个人担惊受怕的想法,我把这个鬼故事讲给了C君,结果C君竟回敬了我一个关于工字厅的鬼故事。
鉴于我们平时就生活在清华园内,这样一个故事某种程度上更加惊悚。

据说那是工字厅门前还不在夜里点灯。
有一次,某位老师下班后又加了很久的班,等他忙完要离开时,天已经完全黑了。
他匆匆收拾好东西就出了门,准备离开。要去开车时,他发现自己竟把车钥匙落在了办公室中,于是又回去拿车钥匙。
然而他这趟回来,跨进工字厅的大门后,却发现原来的办公室不见了,门后是一个从未见过的世界。
% TODO: 可以描写一下这个世界的景象
他吓了一跳,急忙退出来,发现工字厅还是那个工字厅,清华园还是那个清华园;然而再从大门进去,却又是那个异世界。
这道门,已变成了现世与另一个世界的通道。
这位老师不由得心生惶恐,忙从旁边的小侧门绕进工字厅取了钥匙走人了,并发誓从此再也不在工字厅加班。
而就是从这之后,夜里工字厅门前点起了灯,老师们出入也开始只走旁边的小门,说是要对怪力乱神敬而远之。

C君讲罢之时,我们正好骑到二教旁边。
受氛围所动,我又一下子联想起流传甚广的二教鬼故事,身上不禁升起一阵恶寒,于是催促C君迅速离开了这片是非之地。

\paragraph{记事}
6月23日,笔者与C君寻得零零阁后,准备去紫操再叙。骑车途径工字厅,发现工字厅大门在晚上还亮着灯。灯光昏黄,显得有一点点阴森和神秘。
