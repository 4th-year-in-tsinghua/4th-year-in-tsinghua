\chapter{早期建筑群}

我催促着C君赶紧回寝。经过大礼堂时,不留意给骑过了,骑到了大礼堂的后方。
我们正要掉头返回,却惊奇地发现大礼堂后竟然还有着一条小径可以通到大礼堂东侧。
因为天色已晚,我们不想再绕一圈,此时急着回去,也没有多想,便直接骑上了这条小径。

% TODO: 小径的照片

这条小径由行道砖铺就,从大礼堂后的树丛中穿过。要说的话,其实看起来和李文正馆前穿过草坪的几条小道也没什么差别。
但我刚骑上小径,就产生了微妙的奇怪的感觉,总觉得身边似乎有什么东西一样,但是又什么都看不到。我稳了稳心神,暗示自己这只是错觉,然后又专心骑车。
骑到出口时,前面的C君突然一声惊呼,来了个急刹车。我也被他吓了一跳,差点就撞上了他。
“怎么了?”我随口问了一句,然后抬头往前看,就发现有个人正站在路口的树荫下,刚刚C君差点就撞到了他。而那个人也一脸惊异地看着我俩。
我在后面看着那人觉得有些脸熟,向前挪了挪,仔细一看,发现那人是同系的O君。
我感到很奇怪,就问他为什么这么晚了站在这种地方,多么危险。
“我还想说这么晚了,你们怎么从这里出来?我哪里想得到这时候还有人骑车在这里晃荡……所以你们大半夜地在这里晃什么?”他回到。
我告诉他我们去西湖游泳池那里探查游泳池蓄满水的原因。
他听说后脸色一变,忙问我们有没有碰到什么奇怪的现象。
我愣了一下,不知道该不该告诉他西湖游泳池的灵异之物,于是看向C君征求意见。我俩交换了一下眼神,都同意O君可能知道些什么情报,因此决定告诉他此事,便对他细细讲了我们在西湖的见闻。

O君听完后,若有所思地点了点头,然后有问我们穿过这条小道的时候有没有什么奇怪的感觉。
我们一开始没明白他是指什么,愣了一下,然后又突然想起,刚刚似乎确实曾觉得身边好像有什么看不到的东西,于是又对他诉说了当时的感觉。
他听后,又点着头陷入了思索。想了一会儿后,他突然抬起头对我们说:“罢了,反正你们也算是卷入这种事了,我便作为圈内前辈为你们讲解一下吧。”

从O君的讲解中,我们逐渐了解了这片土地上发生过的不为人知的故事。
原来这片老建筑群的区域上存在着古老的结界,保护着历史悠久的冥府之国。
冥府某种意义上属于另一个世界的国度,现世的人一般是看不到他们的。普通人偶尔会无来由地感受到些寒意,那就是其对冥府之民的感知的极限了;只有具有很强的灵感的人,才能够清晰地看到他们,甚至与其交流。而O君就具有很强的灵感,因此能够感受到这里的一切。
结界保护着冥府的居民,隔绝了冥府与外界。但为了保障冥府内部的生活,冥府还要与外界互通有无,故修建结界时,预留了一条通道,这条通道会在每天子时时段打开,这时结界内的居民就可以出来,外面的人也可以进去,相互交易。而这条通道,就是现在看得到的这个大礼堂后的通道。

天色已晚,O君劝我们这两个新手早些回去,改天再带我们去了解其他地方的秘辛。于是我们约定了三天后,由O君带我们游历冥府结界。

\vfill

\paragraph{记事}
6月23日,笔者和C君从零零阁去紫操路上,经过大礼堂,这是笔者四年来第一次发现大礼堂后还有条小道。
刚开始构思时,笔者本打算设定通道开放的时间为晚上7点到9点,因为那天笔者路过此地时是8点多,正当十二时辰中的戌时,即晚上7\~9点。
但是正式开始写之后,笔者构思剧情线时,将路过大礼堂的情节安放在了西湖之后,而剧情中“我们”离开西湖时已接近深夜,故将通道的开放时间改为了深夜子时。
