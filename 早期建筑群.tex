\chapter{早期建筑群}

我催促着C君赶紧回寝。经过大礼堂时,不留意给骑过了,骑到了大礼堂的后方。
我们正要掉头返回,却惊奇地发现大礼堂后竟然还有着一条小径可以通到大礼堂东侧。
因为天色已晚,我们也不想再绕一圈,故直接骑上了这条小径。

这条小径由行道砖铺就,从大礼堂后的树丛中穿过。要说的话,其实看起来和李文正馆前穿过草坪的几条小道也没什么差别。
但我刚骑上小径,就产生了微妙的奇怪的感觉,总觉得身边似乎有什么东西一样,但是又什么都看不到。我稳了稳心神,暗示自己这只是错觉,然后又专心骑车。
马上就要出去这条小道了,骑在前面的C君突然一声惊呼,来了个急刹车。我也被他吓了一跳,差点就撞上了他。
“怎么了?”我随口问了一句,然后抬头往前一看,发现有个人就站在路口的树荫下,刚刚C君差点就撞到了他。而那个人也一脸惊异地看着我俩。
感觉似乎没什么事情,反正最终也没有碰到,于是我们道了声抱歉就准备继续走,这时那个人却开口叫住了我们。

\vfill

\paragraph{记事}
6月23日,笔者和C君从零零阁去紫操路上,经过大礼堂,笔者第一次发现大礼堂后还有条小道。
刚开始构思时,笔者本打算设定通道开放的时间为晚上7点到9点,因为那天笔者路过此地时是8点多,正当十二时辰中的戌时,即晚上7~9点。
但是正式开始写之后,为了构成相对连续的剧情线,因此将路过大礼堂安放在了西湖之后。
而剧情中“我们”离开西湖是接近深夜,因此将通道的开放时间改为了深夜子时。
